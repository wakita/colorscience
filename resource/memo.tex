\documentclass [a4paper,fleqn] {jarticle}

\usepackage {amsmath}

\newcommand\V[1]{\ensuremath {\boldsymbol {#1}}}
\newcommand\M[1]{\ensuremath {\boldsymbol {#1}}}

\begin {document}

\section* {Chapter 1: The Nature of Color}

\subsection* {1.4: Some radiometric and photometric units}

\emph {Irradiance --- $BJ|<M>HEY(B} ($\bar W$) is the total integrated radiant flux for all wavelengths, in watts, per unit area falling on an illuminated body:
\[ \bar W = \frac {\mathrm {d}\phi_R} {\mathrm {d}A}, \]
where $\phi_R$ is for \emph {radiant flux --- $BJ|<MB+(B} and $A$ for area.

The \emph {spectral irradiance --- $BJ,8wJ|<M>HEY(B} ($W(\lambda)$) is related to the irradiance:
\[ \bar W = \int_{-\infty}^\infty W(\lambda) \mathrm {d}\lambda \]

The \emph {radiance --- $BJ|<M51EY(B} ($\bar P$) of a luminous surface:
\[ \bar P = \frac {\text {radiant flux per steradian}} {\text {perceived area}}
          = \frac {1} {cos \theta}
            \frac {\mathrm d^2 \phi_R} {\mathrm d A \mathrm d \Omega} \]

The \emph {spectral radiance --- $BJ,8wJ|<M51EY(B} ($P(\lambda)$), defined as the radiance per unit wavelength interval, is related to the radiance $\bar P$ by:
\[ \bar P = \int_{-\infty}^\infty P(\lambda) \mathrm d\lambda \]

When the light enters the eye, a \emph {luminous stimulus} is produced.  The \emph {luminous flux --- $B8wB+(B} ($\phi_L$) is the equivalent of the radiant flux, but evaluated according to the magnitude of the lumious stimulus it produces, measured in lumens.

\[ \bar L = \frac {\text {luminous flux per steradian}} {\text {perceived area}}
          = \frac {1} {cos \theta}
            \frac {\mathrm d^2 \phi_L} {\mathrm d A \mathrm d \Omega} \]

The \emph {spectral luminance --- $BJ,8w51EY(B} ($\L(\lambda)$) is related to the luminance $\bar L$ by:

\[ \bar L = \int_{-\infty}^\infty L(\lambda) \mathrm d\lambda \]

The \emph {brightness --- $BL@EY(B} is more subjective term than the luminance or the lightness.

The \emph {(spectral) luminous efficiency --- $BH/8w8zN((B} ($\bar \sigma(\lambda)$)\footnote {Spectral luminous efficiency is also known as \emph {luminous sensitivity}.} is defined by the ratio of the (spectral) luminance to the (spectral) radiance:
%
\begin {align*}
  \bar \sigma &= L/P \\
  \sigma(\lambda) &= L(\lambda)/P(\lambda)
\end {align*}

\begin {itemize}
\item Spectral luminous efficiency has a peak value located at a wavelength of 555 nm.
\item One watt of radiant energy at the wavelength of 555 nm is equivalent to 683 lumens.
\end {itemize}

The \emph {relative spectral luminous efficiency --- $BAjBPJ,8wH/8w8zN(!)(B} ($V(\lambda)$) is defined as the luminous efficiency divided by its peak value at a wavelength of 555 nm.
%
\[ \sigma(\lambda) = 683 V(\lambda) \]

\emph {Luminance --- $B51EY(B} of a polychromatic light source ($\bar L$):
%
\[ \bar L = \int_{-\infty}^\infty \sigma(\lambda)P(\lambda) \mathrm d\lambda
          = 683 \int_{-\infty}^\infty V(\lambda)P(\lambda) \mathrm d\lambda \]

The \emph {luminous efficiency of a polychromatic light source} can be shown:
%
\[ \bar \sigma =
   \bar L / \bar R =
   \frac {\int_{-\infty}^\infty \sigma(\lambda)P(\lambda) \mathrm d\lambda}
         {\int_{-\infty}^\infty P(\lambda) \mathrm d\lambda} =
   683
   \frac {\int_{-\infty}^\infty V(\lambda)P(\lambda) \mathrm d\lambda}
         {\int_{-\infty}^\infty P(\lambda) \mathrm d\lambda} \]


\pagebreak

\section* {Chapter 3: Trichromatic Theory}

\subsection* {3.1: Grassmann laws}

\subsection* {3.2: Maxwell triangle}

$r-g-b$

\subsection* {3.3: Color-matching experiments}

To begin the color-matching procedure, the spectrally pure reference field is set to a red color ($\lambda =$ 700nm).  Then the wavelength in the reference field is decreased in constant steps $\Delta \lambda$, preserving the same constant radiance $P_{\mathit {Ref}}$ for all wavelength.

Radiance $L_{\mathit {Ref}}(\lambda)$:
%
\[ L_{\mathit {Ref}}(\lambda) = 683 P_{\mathit {Ref}}V(\lambda). \]

The three luminances, $L_r(\lambda)$, $L_g(\lambda)$, $L_b(\lambda)$.

\begin {align*}
  L_r(\lambda) &= 683 P_r(\lambda)V(\lambda) \\
  L_g(\lambda) &= 683 P_g(\lambda)V(\lambda) \\
  L_b(\lambda) &= 683 P_b(\lambda)V(\lambda)
\end {align*}
%
where
\begin {itemize}
\item $L_r(\lambda)$, $L_g(\lambda)$, $L_b(\lambda)$ are luminances,
\item $P_r(\lambda)$, $P_g(\lambda)$, $P_b(\lambda)$ are radiances, and
\item $V(\lambda)$ is the relative photopic luminous efficiency of the eye.
\end {itemize}

The total luminance of $L_{\mathit {Mch}}(\lambda)$:

\begin {align*}
  L_{\mathit {Mch}}(\lambda)
    &= L_r(\lambda) + L_g(\lambda) + L_b(\lambda)
    &= 683 (P_r(\lambda)V(\lambda) +
            P_g(\lambda)V(\lambda) +
            P_b(\lambda)V(\lambda))
\end {align*}

\subsection* {4.1 The CIE color system}

\[ \V {xyz} = \M {A} \cdot \V {rgb} \]

\subsection* {4.2 Color-matching functions $\bar {x}(\lambda)$, $\bar {y}(\lambda)$, $\bar {z}(\lambda)$}

\begin {enumerate}

\item The function $\bar y(\lambda)$ should be equal to the luminosity function $V(\lambda)$ of the eye.
%
\[ \bar y(\lambda) = V(\lambda) \]

\item Constant energy of white, $P(\lambda) = 1$, should have equal values for the three tristimuli.

\item The line joining points X and Y in Figure 3.8 be tangent to the curve on the red side.

\item $\bar y$-axis must be tangent to the curve, so that no value of $\bar x(\lambda)$ is negative.

\end {enumerate}

\[ \V{xyz} =
   \begin {pmatrix}
     2.76888 & 1.75175 & 1.13016 \\
     1       & 4.59070 & 0.06010 \\
     0       & 0.05651 & 5.59427
   \end {pmatrix} \V{rgb} \]

\subsection* {4.3 Tristimulus values $X$, $Y$, $Z$}

\[ \V{XYZ} = \int_0^\infty P(\lambda)\V {xyz}(\lambda) \mathrm {d}\lambda \]

\end {document}
